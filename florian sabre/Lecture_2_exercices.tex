% Options for packages loaded elsewhere
\PassOptionsToPackage{unicode}{hyperref}
\PassOptionsToPackage{hyphens}{url}
%
\documentclass[
]{article}
\usepackage{lmodern}
\usepackage{amssymb,amsmath}
\usepackage{ifxetex,ifluatex}
\ifnum 0\ifxetex 1\fi\ifluatex 1\fi=0 % if pdftex
  \usepackage[T1]{fontenc}
  \usepackage[utf8]{inputenc}
  \usepackage{textcomp} % provide euro and other symbols
\else % if luatex or xetex
  \usepackage{unicode-math}
  \defaultfontfeatures{Scale=MatchLowercase}
  \defaultfontfeatures[\rmfamily]{Ligatures=TeX,Scale=1}
\fi
% Use upquote if available, for straight quotes in verbatim environments
\IfFileExists{upquote.sty}{\usepackage{upquote}}{}
\IfFileExists{microtype.sty}{% use microtype if available
  \usepackage[]{microtype}
  \UseMicrotypeSet[protrusion]{basicmath} % disable protrusion for tt fonts
}{}
\makeatletter
\@ifundefined{KOMAClassName}{% if non-KOMA class
  \IfFileExists{parskip.sty}{%
    \usepackage{parskip}
  }{% else
    \setlength{\parindent}{0pt}
    \setlength{\parskip}{6pt plus 2pt minus 1pt}}
}{% if KOMA class
  \KOMAoptions{parskip=half}}
\makeatother
\usepackage{xcolor}
\IfFileExists{xurl.sty}{\usepackage{xurl}}{} % add URL line breaks if available
\IfFileExists{bookmark.sty}{\usepackage{bookmark}}{\usepackage{hyperref}}
\hypersetup{
  pdftitle={Exercises lecture 3 -- data wrangling},
  pdfauthor={Paolo Crosetto},
  hidelinks,
  pdfcreator={LaTeX via pandoc}}
\urlstyle{same} % disable monospaced font for URLs
\usepackage[margin=1in]{geometry}
\usepackage{color}
\usepackage{fancyvrb}
\newcommand{\VerbBar}{|}
\newcommand{\VERB}{\Verb[commandchars=\\\{\}]}
\DefineVerbatimEnvironment{Highlighting}{Verbatim}{commandchars=\\\{\}}
% Add ',fontsize=\small' for more characters per line
\usepackage{framed}
\definecolor{shadecolor}{RGB}{248,248,248}
\newenvironment{Shaded}{\begin{snugshade}}{\end{snugshade}}
\newcommand{\AlertTok}[1]{\textcolor[rgb]{0.94,0.16,0.16}{#1}}
\newcommand{\AnnotationTok}[1]{\textcolor[rgb]{0.56,0.35,0.01}{\textbf{\textit{#1}}}}
\newcommand{\AttributeTok}[1]{\textcolor[rgb]{0.77,0.63,0.00}{#1}}
\newcommand{\BaseNTok}[1]{\textcolor[rgb]{0.00,0.00,0.81}{#1}}
\newcommand{\BuiltInTok}[1]{#1}
\newcommand{\CharTok}[1]{\textcolor[rgb]{0.31,0.60,0.02}{#1}}
\newcommand{\CommentTok}[1]{\textcolor[rgb]{0.56,0.35,0.01}{\textit{#1}}}
\newcommand{\CommentVarTok}[1]{\textcolor[rgb]{0.56,0.35,0.01}{\textbf{\textit{#1}}}}
\newcommand{\ConstantTok}[1]{\textcolor[rgb]{0.00,0.00,0.00}{#1}}
\newcommand{\ControlFlowTok}[1]{\textcolor[rgb]{0.13,0.29,0.53}{\textbf{#1}}}
\newcommand{\DataTypeTok}[1]{\textcolor[rgb]{0.13,0.29,0.53}{#1}}
\newcommand{\DecValTok}[1]{\textcolor[rgb]{0.00,0.00,0.81}{#1}}
\newcommand{\DocumentationTok}[1]{\textcolor[rgb]{0.56,0.35,0.01}{\textbf{\textit{#1}}}}
\newcommand{\ErrorTok}[1]{\textcolor[rgb]{0.64,0.00,0.00}{\textbf{#1}}}
\newcommand{\ExtensionTok}[1]{#1}
\newcommand{\FloatTok}[1]{\textcolor[rgb]{0.00,0.00,0.81}{#1}}
\newcommand{\FunctionTok}[1]{\textcolor[rgb]{0.00,0.00,0.00}{#1}}
\newcommand{\ImportTok}[1]{#1}
\newcommand{\InformationTok}[1]{\textcolor[rgb]{0.56,0.35,0.01}{\textbf{\textit{#1}}}}
\newcommand{\KeywordTok}[1]{\textcolor[rgb]{0.13,0.29,0.53}{\textbf{#1}}}
\newcommand{\NormalTok}[1]{#1}
\newcommand{\OperatorTok}[1]{\textcolor[rgb]{0.81,0.36,0.00}{\textbf{#1}}}
\newcommand{\OtherTok}[1]{\textcolor[rgb]{0.56,0.35,0.01}{#1}}
\newcommand{\PreprocessorTok}[1]{\textcolor[rgb]{0.56,0.35,0.01}{\textit{#1}}}
\newcommand{\RegionMarkerTok}[1]{#1}
\newcommand{\SpecialCharTok}[1]{\textcolor[rgb]{0.00,0.00,0.00}{#1}}
\newcommand{\SpecialStringTok}[1]{\textcolor[rgb]{0.31,0.60,0.02}{#1}}
\newcommand{\StringTok}[1]{\textcolor[rgb]{0.31,0.60,0.02}{#1}}
\newcommand{\VariableTok}[1]{\textcolor[rgb]{0.00,0.00,0.00}{#1}}
\newcommand{\VerbatimStringTok}[1]{\textcolor[rgb]{0.31,0.60,0.02}{#1}}
\newcommand{\WarningTok}[1]{\textcolor[rgb]{0.56,0.35,0.01}{\textbf{\textit{#1}}}}
\usepackage{graphicx,grffile}
\makeatletter
\def\maxwidth{\ifdim\Gin@nat@width>\linewidth\linewidth\else\Gin@nat@width\fi}
\def\maxheight{\ifdim\Gin@nat@height>\textheight\textheight\else\Gin@nat@height\fi}
\makeatother
% Scale images if necessary, so that they will not overflow the page
% margins by default, and it is still possible to overwrite the defaults
% using explicit options in \includegraphics[width, height, ...]{}
\setkeys{Gin}{width=\maxwidth,height=\maxheight,keepaspectratio}
% Set default figure placement to htbp
\makeatletter
\def\fps@figure{htbp}
\makeatother
\setlength{\emergencystretch}{3em} % prevent overfull lines
\providecommand{\tightlist}{%
  \setlength{\itemsep}{0pt}\setlength{\parskip}{0pt}}
\setcounter{secnumdepth}{-\maxdimen} % remove section numbering

\title{Exercises lecture 3 -- data wrangling}
\author{Paolo Crosetto}
\date{octobre 2020}

\begin{document}
\maketitle

\hypertarget{filter-select}{%
\section{\texorpdfstring{\texttt{filter()} \&
\texttt{select()}}{filter() \& select()}}\label{filter-select}}

\hypertarget{exercice-1}{%
\subsection{Exercice 1}\label{exercice-1}}

\begin{quote}
sauvegardez dans un nouvel objet tous les vols partis entre midi et deux
heures, en gardant juste l'info sur l'aéroport de départ et d'arrivée
\end{quote}

\begin{Shaded}
\begin{Highlighting}[]
\NormalTok{df }\OperatorTok\StringTok{ }
\StringTok{  }\KeywordTok{filter}\NormalTok{(dep_time }\OperatorTok{>=}\StringTok{ }\DecValTok{1200} \OperatorTok{&}\StringTok{ }\NormalTok{dep_time }\OperatorTok{<=}\StringTok{ }\DecValTok{1400}\NormalTok{) }\OperatorTok\StringTok{ }
\StringTok{  }\KeywordTok{select}\NormalTok{(dep_time, origin, dest) ->}\StringTok{ }\NormalTok{df_midday}
\end{Highlighting}
\end{Shaded}

\hypertarget{exercice-2}{%
\subsection{Exercice 2}\label{exercice-2}}

\begin{quote}
isolez dans un nouvel objet tous les vols partis entre minuit et une
heure du matin de JFK et de LGA. Quelle est, pour chacun de deux
aéroport, la destionation la plus fréquente?
\end{quote}

\begin{Shaded}
\begin{Highlighting}[]
\NormalTok{df}\OperatorTok
\StringTok{  }\KeywordTok{filter}\NormalTok{(dep_time}\OperatorTok{>=}\DecValTok{0000} \OperatorTok{&}\StringTok{ }\NormalTok{dep_time}\OperatorTok{<=}\DecValTok{0100}\NormalTok{)}
\end{Highlighting}
\end{Shaded}

\begin{verbatim}
## # A tibble: 892 x 19
##     year month   day dep_time sched_dep_time dep_delay arr_time sched_arr_time
##    <int> <int> <int>    <int>          <int>     <dbl>    <int>          <int>
##  1  2013     1     2       42           2359        43      518            442
##  2  2013     1     3       32           2359        33      504            442
##  3  2013     1     3       50           2145       185      203           2311
##  4  2013     1     4       25           2359        26      505            442
##  5  2013     1     5       14           2359        15      503            445
##  6  2013     1     5       37           2230       127      341            131
##  7  2013     1     6       16           2359        17      451            442
##  8  2013     1     7       49           2359        50      531            444
##  9  2013     1     9        2           2359         3      432            444
## 10  2013     1     9        8           2359         9      432            437
## # ... with 882 more rows, and 11 more variables: arr_delay <dbl>,
## #   carrier <chr>, flight <int>, tailnum <chr>, origin <chr>, dest <chr>,
## #   air_time <dbl>, distance <dbl>, hour <dbl>, minute <dbl>, time_hour <dttm>
\end{verbatim}

\begin{Shaded}
\begin{Highlighting}[]
  \KeywordTok{filter}\NormalTok{(df,origin}\OperatorTok{==}\StringTok{'JFK'} \OperatorTok{&}\StringTok{ }\NormalTok{origin}\OperatorTok{==}\StringTok{'LGA'}\NormalTok{)->}\StringTok{ }\NormalTok{OBJ1}
\end{Highlighting}
\end{Shaded}

\hypertarget{mutate}{%
\section{\texorpdfstring{\texttt{mutate()}}{mutate()}}\label{mutate}}

\hypertarget{exercice-3}{%
\subsection{Exercice 3}\label{exercice-3}}

\begin{quote}
créez une variable qui montre la vitesse de chaque avion
\end{quote}

\hypertarget{exercice-4}{%
\subsection{Exercice 4}\label{exercice-4}}

\begin{quote}
créez une variable qui calcule l'impact (en \%) du retard à l'arrivée
sur le temps de vol
\end{quote}

\hypertarget{summarise-and-group_by}{%
\section{\texorpdfstring{\texttt{summarise()} and
\texttt{group\_by()}}{summarise() and group\_by()}}\label{summarise-and-group_by}}

\hypertarget{exercice-5}{%
\subsection{Exercice 5}\label{exercice-5}}

\begin{quote}
calculez la moyenne, l'écart type, le min et le max du rétard à
l'arrivée
\end{quote}

\hypertarget{exercice-6}{%
\subsection{Exercice 6}\label{exercice-6}}

\begin{quote}
même chose que l'exercice 6, mais par aéroport de départ
\end{quote}

\hypertarget{exercice-7}{%
\subsection{Exercice 7}\label{exercice-7}}

\begin{quote}
calculez la moyenne du retard par compagnie aérienne
\end{quote}

\hypertarget{exercice-8-filter-select-mutate-summarise-group_by}{%
\subsection{Exercice 8 -- filter + select + mutate + summarise +
group\_by}\label{exercice-8-filter-select-mutate-summarise-group_by}}

\begin{quote}
quelle est la vitesse moyenne des vols qui partent entre 11h et 13h, par
mois?
\end{quote}

\hypertarget{meet-the-pipe}{%
\section{\texorpdfstring{meet the pipe:
\texttt{\%\textgreater{}\%}}{meet the pipe: \%\textgreater\%}}\label{meet-the-pipe}}

\hypertarget{meta-exercice-1}{%
\subsection{meta-exercice 1}\label{meta-exercice-1}}

\begin{quote}
\textbf{re-faites} \emph{tous} les exercices ci-dessus en utilisant
l'opérateur `et après' / pipe \texttt{\%\textgreater{}\%}
\end{quote}

\hypertarget{exercice-9}{%
\subsection{Exercice 9}\label{exercice-9}}

\begin{quote}
trouvez le maximum retard au depart par aéroport pour JFK et LGA pour
chaque jour de l'an. Est-ce que les retards sont corrélés?
\end{quote}

\hypertarget{exercice-10}{%
\subsection{Exercice 10}\label{exercice-10}}

\begin{quote}
de quel aéroport partent les vols à plus longue distance?
\end{quote}

\hypertarget{join_...-family-of-functions}{%
\section{\texorpdfstring{\texttt{join\_...()} family of
functions}{join\_...() family of functions}}\label{join_...-family-of-functions}}

\begin{quote}
first run this setup R code chunk. It will load in your workspace 3 data
frames:
\end{quote}

\begin{itemize}
\tightlist
\item
  \textbf{airports}: avec données sur les aéroports américains
\item
  \textbf{flights}: qu'on connait déjà
\item
  \textbf{planes}: avec les données pour chaque avion
\end{itemize}

\begin{Shaded}
\begin{Highlighting}[]
\NormalTok{planes <-}\StringTok{ }\NormalTok{planes}
\NormalTok{flights <-}\StringTok{ }\NormalTok{flights}
\NormalTok{airports <-}\StringTok{ }\NormalTok{airports}
\end{Highlighting}
\end{Shaded}

\hypertarget{exercice-11}{%
\subsection{Exercice 11}\label{exercice-11}}

\begin{quote}
est-ce que les routes plus longues sont desservies apr les avions les
plus modernes?
\end{quote}

\emph{notes}: utilisez \texttt{left\_join()} et mergez les dataframes
\texttt{flights} et \texttt{planes}

\hypertarget{exercice-12}{%
\subsection{Exercice 12}\label{exercice-12}}

\begin{quote}
combien de vols qui partent des trois aéroport de NY atterrisent dans
des destinations au dessus de 1000m s.n.m.?
\end{quote}

\hypertarget{creating-tidy-data-reshape-with-gather-and-spread}{%
\section{\texorpdfstring{creating tidy data: reshape with
\texttt{gather()} and
\texttt{spread()}}{creating tidy data: reshape with gather() and spread()}}\label{creating-tidy-data-reshape-with-gather-and-spread}}

\hypertarget{exercice-13}{%
\subsection{Exercice 13}\label{exercice-13}}

\begin{quote}
tidy world\_bank\_pop dataset so that `year' is a variable and for each
country and each year you have urban population and urban population
growh only. Plot as a geom\_line the total population for each country
over the years.
\end{quote}

\end{document}
